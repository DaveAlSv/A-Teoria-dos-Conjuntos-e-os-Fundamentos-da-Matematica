\documentclass[portuguese,12pt,a4paper]{book}
\usepackage[utf8]{inputenc}
\usepackage[T1]{fontenc}
\usepackage{graphicx}
\usepackage{mathtools}
\usepackage{amssymb}
\usepackage{amsthm}
\usepackage[colorlinks=true]{hyperref}
\usepackage{enumitem}
\usepackage[
backend=biber,
style=alphabetic,
sorting=ynt
]{biblatex}
\addbibresource{bibliografia.bib}

\title{A teoria dos conjuntos e os fundamentos da matemática - Soluções}

\newcommand{\cqd}{\hfill $\square$}

\newenvironment{solucao}[1][]{\noindent\textbf{Solução:} }{\cqd}
\newcounter{ex}
\newtheorem{exercicio}[ex]{Exercício}

\begin{document}

    \author{}
	\maketitle
    \tableofcontents
	\chapter{Hist�ria e Motiva��o}

\begin{exercicio}
	Exiba uma bije��o entre o conjunto dos n�meros inteiros e os naturais.
\end{exercicio}
\begin{solucao}
	Defina a fun��o $f:\mathbb{N}\to\mathbb{Z}$ por $f(n)=-\frac{n}{2}$ se $n$ � par e por $f(n)=\frac{n+1}{2}$ se $n$ � �mpar. $f$ � uma bije��o.
\end{solucao}

\begin{exercicio}
	Prove que qualquer subconjunto infinito dos n�meros naturais � enumer�vel.
\end{exercicio}
\begin{solucao}
	Lembre: dizemos que um conjunto $X$ � \textit{enumer�vel} se � finito ou � equipotente a $\mathbb{N}$, isto �, se existe uma bije��o $f:\mathbb{N}\to X$.
	
	Todo conjunto finito � enumer�vel. Est�o vamos olhar apenas o caso $X$ ser infinito. Defina $f:\mathbb{N}\to X$ da seguinte maneira: $f(0)$ � o menor elemento de $X$; $f(1)$ � o segundo menor elemento de $X$; $f(2)$ � o terceiro menor elemento de $X$, e assim por diante. De maneira mais rigorosa, $f$ � definida indutivamente por: $f(0)=\min X$; uma vez escolhidos, $f(0),f(1),\ldots,f(n)$, definimos $f(n+1)=\min(X\backslash\{f(0),f(1),\ldots,f(n)\})$. Observe que $f$ � crescente, logo injetiva. Tamb�m � sobrejetiva, pois do contr�rio, se existe um $x\in X\backslash f(\mathbb{N})$, ent�o $x\in X\backslash\{f(0),f(1),\ldots,f(n)\}$ para cada $n\in\mathbb{N}$. Da maneira como constru�mos $f$ conclu�mos que $x>f(n)\ \forall n\in\mathbb{N}$ e assim $f(\mathbb{N})$ � um conjunto limitado, logo finito. Como $f:\mathbb{N}\to f(\mathbb{N})$ � uma bije��o, $\mathbb{N}$ seria finito, um absurdo. Portanto $f:\mathbb{N}\to X$ � uma bije��o, logo $X$ � enumer�vel.
\end{solucao}

\begin{exercicio}
	Na bije��o que constru�mos entre os n�meros naturais e os polin�mios, encontre o polin�mio associado ao n�mero 30.
\end{exercicio}
\begin{solucao}
	\begin{equation*}
		\begin{array}{|ccc|ccc|ccc|}
			\hline
			0  & \mapsto & -x-1  & 11 & \mapsto & -x-2       & 21 & \mapsto & -2x^2-2x-1 \\ \hline
			1  & \mapsto & -x    & 12 & \mapsto & -x+2       & 22 & \mapsto & -2x^2-2x   \\ \hline
			2  & \mapsto & -x+1  & 13 & \mapsto & x-2        & 23 & \mapsto & -2x^2-2x+1 \\ \hline
			3  & \mapsto & x-1   & 14 & \mapsto & x+2        & 24 & \mapsto & -2x^2-2x+2 \\ \hline
			4  & \mapsto & x     & 15 & \mapsto & 2x-2       & 25 & \mapsto & -2x^2-x-2  \\ \hline
			5  & \mapsto & x+1   & 16 & \mapsto & 2x-1       & 26 & \mapsto & -2x^2-x-1  \\ \hline
			6  & \mapsto & -2x-2 & 17 & \mapsto & 2x         & 27 & \mapsto & -2x^2-x    \\ \hline
			7  & \mapsto & -2x-1 & 18 & \mapsto & 2x+1       & 28 & \mapsto & -2x^2-x+1  \\ \hline
			8  & \mapsto & -2x   & 19 & \mapsto & 2x+2       & 29 & \mapsto & -2x^2-x+2  \\ \hline
			9  & \mapsto & -2x+1 & 20 & \mapsto & -2x^2-2x-2 & 30 & \mapsto & -2x^2-2    \\ \hline
			10 & \mapsto & -2x+2 &    &         &            &    &         &            \\ \hline
		\end{array}
	\end{equation*}
	Portanto o polin�mio associado ao n�mero 30 � $-2x^2-2$.
\end{solucao}

\begin{exercicio}
	Na bije��o que constru�mos entre os n�meros naturais e os n�meros alg�bricos, encontre o n�mero natural associado ao n�mero $\sqrt{3}$
\end{exercicio}
\begin{solucao}
	content
\end{solucao}

\begin{exercicio}
	Suponha que, em um conjunto infinito, existe uma forma de representar cada elemento do conjunto com uma sequ�ncia finita de s�mbolos, dentre um conjunto finito de s�mbolos. Mostre que esse conjunto � enumer�vel e use esse resultado diretamente para mostrar que os conjuntos dos n�meros racionais e dos n�meros alg�bricos s�o enumer�veis.
\end{exercicio}
\begin{solucao}
	Seja $\Sigma=\{\sigma_1,\ldots,\sigma_m\}$ o conjunto de s�mbolos. Para cada $n\in\mathbb{N}$, seja $S_n=\{(\tau_1,\ldots,\tau_n):\tau_1,\ldots,\tau_n\in \Sigma\}$ e ponha $S=\bigcup_{n=1}^\infty S_n$. $S$ � enumer�vel por ser uma reuni�o enumer�vel de conjuntos enumer�veis (finitos).\footnote{\cite[p. 51]{Lima16}} Seja $X$ o conjunto infinito do exerc�cio. A suposi��o de que cada elemento de $X$ possa ser representado como uma sequ�ncia finita de s�mbolos significa que existe uma fun��o $f:X\to S$. Esta fun��o deve ser injetiva, pois n�o podemos usar a mesma sequ�ncia de s�mbolos para representar dois ou mais elementos distintos em $X$. Ent�o $f: X\to f(X)\subset S$ � bijetiva e como $f(X)$ � enumer�vel, por ser um subconjunto de um conjunto enumer�vel conforme o Exerc�cio 2, $X$ � enumer�vel.
	
	Todo n�mero racional pode ser escrito como $a/b$, com $a,b\in \mathbb{Z}$, $b\neq 0$. Ambos $a$ e $b$ s�o escritos como uma sequ�ncia finita de s�mbolos (os algarismos de 0 a 9 e o sinal $+$ ou $-$). Pelo resultado demonstrado acima, $\mathbb{Q}$ � enumer�vel.
	
	Cada polin�mio com coeficientes inteiros pode ser escrito como uma sequ�ncia finita de s�mbolos, logo o conjunto formado por esses polin�mios � enumer�vel. O conjunto dos n�meros alg�bricos � a uni�o dos conjuntos das ra�zes de cada um desses polin�mios. Como essa � uma uni�o enumer�vel (o conjunto desses polin�mios � enumer�vel) de conjuntos finitos (cada polin�mio possui um n�mero finito de ra�zes), o conjunto dos n�meros alg�bricos � enumer�vel.
\end{solucao}

\begin{exercicio}
	Imagine que o hotel de Hilbert, com uma quantidade infinita enumer�vel de quartos, todos ocupados, receba infinitos trens com infinitos vag�es e cada vag�o com infinitos passageiros (todas essas quantidades enumer�veis). Como o gerente pode alocar todos os atuais h�spedes em quartos separados?
\end{exercicio}
\begin{solucao}
	Cada passageiro tem uma identifica��o $(p,q,r)$: $p$ = n�mero do trem; $q$ = n�mero do vag�o; $r$ = n�mero do assento. Cada h�spede no quarto $n$ ser� realocado para o quarto $2n-1$, e cada passageiro com ID $(p,q,r)$ ser� hospedado no quarto $2^p3^q(2r-1)$.
	
	Isto pode ser generalizado: h�spede $n\mapsto$ quarto $2n-1$; passageiro $(a_1,\ldots,a_m)\mapsto$ quarto $2^{a_1}3^{a_2}\cdots p_{m-1}^{a_{m-1}}(2a_m-1)$, em que $2,3,\ldots,p_{m-1}$ s�o n�meros primos.
\end{solucao}

\begin{exercicio}
	Imagine, agora, um hotem maior ainda, com um quarto para cada n�mero real, totalmente ocupado. Um �nibus igualmente gigantesco, com um passageiro para cada n�mero real, chega ao hotel. Como o gerente pode fazer para rearranjar os h�spedes para acolher os novos visitantes, sempre em quartos separados?
\end{exercicio}
\begin{solucao}
	\begin{itemize}
		\item H�spede $x\mapsto$ quarto $\arctan x$.
		\item Passageiro $y\mapsto$ quarto $y+\pi/2$ se $y\geq0$ ou $y-\pi/2$ se $y<0$. 
	\end{itemize}
	Note que o hotel ficar� com um quarto vago (o de n�mero $-\pi/2$).
\end{solucao}
	\chapter{A linguagem da Teoria dos Conjuntos}

\begin{exercicio}
	Usando a linguagem de primeira ordem da teoria de conjuntos, escreva fórmulas para representar as seguintes frases.
\end{exercicio}

\begin{solucao}
\begin{enumerate}
	\item[a)] Não existe o conjunto de todos os conjuntos.
    \item[] $\nexists x \forall y (y \in x) \equiv \forall x \exists y (y \notin x)$
	\item[b)] Existe um único conjunto vazio.
    \item[] $\exists ! x \forall y (y \notin x)$
	\item[c)] x é um conjunto unitário
    \item[] $\exists ! y (y \in x)$
	\item[d)] Existe um conjunto que tem como elemento apenas o conjunto vazio
    \item[] $\exists x \forall y ((y \in x) \leftrightarrow y = \phi)$
	\item[e)] y é o conjunto dos subconjuntos de x 
    \item[] 1 ) $r \text{ é subconjunto de } x \equiv \forall a ((a \in r) \rightarrow (a \in x)) = A_r$, então podemos escrever $\exists y \forall w (A_r^w \leftrightarrow w \in y)$
    \item[] 2 ) $ \exists y \forall z (z \in y \leftrightarrow z \subseteq x )$
    
\end{enumerate}

\end{solucao}

\begin{exercicio}
	Marque as ocorrências de variáveis livres nas fórmulas abaixo
\end{exercicio}

\begin{enumerate}
	\item[a)] $(\forall x (x=y)) \rightarrow (x \in y ) $
    \item[] $x$ e $y$
	\item[b)] $ \forall x ((x=y) \rightarrow (x \in y))$
    \item[] $y$
	\item[c)] $\forall x(x=x) \rightarrow (\forall y \exists Z ((x=y) \land (y=z)) \rightarrow \neg(x\in y))$
    \item[] $x$
	\item[d)] $ \forall x \exists y(\neg(x=y) \land \forall z ((x \in y) \leftrightarrow \forall w ((w \in z ) \rightarrow (w \in x )))) $
    \item[] Não há variáveis livres.
	\item[e)] $(x=y)\rightarrow \exists (x=y) $
    \item[] $x$ e $y$
\end{enumerate}

\begin{exercicio}
	Escreva as subfórmulas de cada fórmula do exercício 2.
\end{exercicio}

\begin{solucao}

\begin{enumerate}
	\item 
	\begin{enumerate}
		\item $(\forall x (x = y)) \rightarrow (x \in y)$
		\item $(\forall x (x = y))$
		\item $(x = y)$
		\item $(x \in y)$
	\end{enumerate}
	
	\item 
	\begin{enumerate}
		\item $\forall x ((x = y) \rightarrow (x \in y))$
		\item $(x = y) \rightarrow (x \in y)$
		\item $(x = y)$
		\item $(x \in y)$
	\end{enumerate}
	
	\item 
	\begin{enumerate}
		\item $\forall x (x = x) \rightarrow (\forall y \exists z (((x = y) \land (y = z)) \rightarrow \neg (x \in y)))$
		\item $\forall x (x = x)$
		\item $(x = y)$
		\item $\forall z \exists y (((x = y) \land (y = z)) \rightarrow \neg (x \in y))$
		\item $((x = y) \land (y = z))$
		\item $(x = y)$
		\item $(y = z)$
		\item $\neg (x \in y)$
		\item $(x \in y)$
	\end{enumerate}
	
	\item 
	\begin{enumerate}
		\item $(x = y) \rightarrow \exists y (x = y)$
		\item $(x = y)$
		\item $\exists y (x = y)$
		\item $(x = y)$
	\end{enumerate}
\end{enumerate}

\end{solucao}
	\chapter{Primeiros Axiomas}
	\chapter{Produto Cartesiano, Relações e Funções}
	\chapter{Axioma da Escolha e suas Aplica��es}
	\chapter{Conjuntos Equipotentes}
	\chapter{Ordinais}
	\chapter{Cardinais}
	\chapter{Ordens Parciais}
	\chapter{Noções de Teoria dos Modelos}
	\chapter{Modelos para ZFC}
	\chapter{\emph{Forcing}}
	
	\printbibliography

\end{document}
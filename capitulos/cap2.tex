\chapter{A linguagem da Teoria dos Conjuntos}

\begin{exercicio}
	Usando a linguagem de primeira ordem da teoria de conjuntos, escreva fórmulas para representar as seguintes frases.
\end{exercicio}

\begin{solucao}
\begin{enumerate}
	\item[a)] Não existe o conjunto de todos os conjuntos.
    \item[] $\nexists x \forall y (y \in x) \equiv \forall x \exists y (y \notin x)$
	\item[b)] Existe um único conjunto vazio.
    \item[] $\exists ! x \forall y (y \notin x)$
	\item[c)] x é um conjunto unitário
    \item[] $\exists ! y (y \in x)$
	\item[d)] Existe um conjunto que tem como elemento apenas o conjunto vazio
    \item[] $\exists x \forall y ((y \in x) \leftrightarrow y = \phi)$
	\item[e)] y é o conjunto dos subconjuntos de x 
    \item[] 1 ) $r \text{ é subconjunto de } x \equiv \forall a ((a \in r) \rightarrow (a \in x)) = A_r$, então podemos escrever $\exists y \forall w (A_r^w \leftrightarrow w \in y)$
    \item[] 2 ) $ \exists y \forall z (z \in y \leftrightarrow z \subseteq x )$
    
\end{enumerate}

\end{solucao}

\begin{exercicio}
	Marque as ocorrências de variáveis livres nas fórmulas abaixo
\end{exercicio}

\begin{enumerate}
	\item[a)] $(\forall x (x=y)) \rightarrow (x \in y ) $
    \item[] $x$ e $y$
	\item[b)] $ \forall x ((x=y) \rightarrow (x \in y))$
    \item[] $y$
	\item[c)] $\forall x(x=x) \rightarrow (\forall y \exists Z ((x=y) \land (y=z)) \rightarrow \neg(x\in y))$
    \item[] $x$
	\item[d)] $ \forall x \exists y(\neg(x=y) \land \forall z ((x \in y) \leftrightarrow \forall w ((w \in z ) \rightarrow (w \in x )))) $
    \item[] Não há variáveis livres.
	\item[e)] $(x=y)\rightarrow \exists (x=y) $
    \item[] $x$ e $y$
\end{enumerate}

\begin{exercicio}
	Escreva as subfórmulas de cada fórmula do exercício 2.
\end{exercicio}

\begin{solucao}

\begin{enumerate}
	\item 
	\begin{enumerate}
		\item $(\forall x (x = y)) \rightarrow (x \in y)$
		\item $(\forall x (x = y))$
		\item $(x = y)$
		\item $(x \in y)$
	\end{enumerate}
	
	\item 
	\begin{enumerate}
		\item $\forall x ((x = y) \rightarrow (x \in y))$
		\item $(x = y) \rightarrow (x \in y)$
		\item $(x = y)$
		\item $(x \in y)$
	\end{enumerate}
	
	\item 
	\begin{enumerate}
		\item $\forall x (x = x) \rightarrow (\forall y \exists z (((x = y) \land (y = z)) \rightarrow \neg (x \in y)))$
		\item $\forall x (x = x)$
		\item $(x = y)$
		\item $\forall z \exists y (((x = y) \land (y = z)) \rightarrow \neg (x \in y))$
		\item $((x = y) \land (y = z))$
		\item $(x = y)$
		\item $(y = z)$
		\item $\neg (x \in y)$
		\item $(x \in y)$
	\end{enumerate}
	
	\item 
	\begin{enumerate}
		\item $(x = y) \rightarrow \exists y (x = y)$
		\item $(x = y)$
		\item $\exists y (x = y)$
		\item $(x = y)$
	\end{enumerate}
\end{enumerate}

\end{solucao}
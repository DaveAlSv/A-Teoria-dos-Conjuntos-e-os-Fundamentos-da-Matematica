\documentclass{article}
\usepackage[utf8]{inputenc}
\usepackage{amsmath}

\begin{document}

\section*{Capítulo 2}

\subsection*{Exercício 1}

\begin{enumerate}
    \item[a)] $\neg \exists x \forall y (y \in x) \equiv \forall x \exists y (y \notin x)$
    \item[b)] $\exists ! x \forall y (y \notin x)$
    \item[c)] $\exists ! y (y \in x)$
    \item[d)] $\exists x \forall y ((y \in x) \rightarrow y = \phi)$
    \item[e)] $r \text{ é subconjunto de } x \equiv \forall a ((a \in r) \rightarrow (a \in x)) \equiv A$, então podemos escrever $\forall w ([A]_r^w \rightarrow w \in y)$
\end{enumerate}

\subsection*{Exercício 2}

\begin{enumerate}
    \item[a)] $y$
    \item[b)] $y$
    \item[c)] $x$
    \item[d)] Não há variáveis livres.
    \item[e)] $x$ e $y$
 \end{enumerate}
 
\subsection*{Exercício 3}

\begin{enumerate}
    \item 
    \begin{enumerate}
        \item $(\forall x (x = y)) \rightarrow (x \in y)$
        \item $(\forall x (x = y))$
        \item $(x = y)$
        \item $(x \in y)$
    \end{enumerate}

    \item 
    \begin{enumerate}
        \item $\forall x ((x = y) \rightarrow (x \in y))$
        \item $(x = y) \rightarrow (x \in y)$
        \item $(x = y)$
        \item $(x \in y)$
    \end{enumerate}

    \item 
    \begin{enumerate}
        \item $\forall x (x = x) \rightarrow (\forall y \exists z (((x = y) \land (y = z)) \rightarrow \neg (x \in y)))$
        \item $\forall x (x = x)$
        \item $(x = y)$
        \item $\forall z \exists y (((x = y) \land (y = z)) \rightarrow \neg (x \in y))$
        \item $((x = y) \land (y = z))$
        \item $(x = y)$
        \item $(y = z)$
        \item $\neg (x \in y)$
        \item $(x \in y)$
    \end{enumerate}

    \item 
    \begin{enumerate}
        \item $(x = y) \rightarrow \exists y (x = y)$
        \item $(x = y)$
        \item $\exists y (x = y)$
        \item $(x = y)$
    \end{enumerate}
\end{enumerate}

\end{document}